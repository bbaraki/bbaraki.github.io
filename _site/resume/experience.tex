
%EXPERIENCE
% \section{\textbf{Experience}}
% \begin{tabularx}{\linewidth}{ @{}l r@{} }
% \textbf{University of Pennsylvania} - Philadelphia, PA. \\[2pt]
% \textbf{Robotics Researcher} - Haptic Exploration \hfill \textbf{July 2024 - Present} \\[2pt]
% \begin{minipage}[t]{\linewidth}
%     \begin{itemize}[nosep,after=\strut, leftmargin=1em, itemsep=2pt]
%         \item  Developed a tactile sensing system for a multi-fingered robot, enabling characterization of objects through haptic feedback
%         \item  Created a data processing pipeline for an optical tactile sensor for real-time force estimation
%         \item Developed a haptic exploration algorithm using a learned dynamic model, capable of predicting discontinuous contact impulses
%     \end{itemize}
% \end{minipage}
% \end{tabularx}

% \begin{tabularx}{\linewidth}{ @{}l r@{} }
% \textbf{Robotics Researcher} - Robotic Swimmer \hfill \textbf{Feb. 2021 - May 2023} \\[2pt]
% \begin{minipage}[t]{\linewidth}
%     \begin{itemize}[nosep,after=\strut, leftmargin=1em, itemsep=2pt]
%         \item Improved the design of a soft-bodied robotic swimmer that utilizes jet propulsion for locomotion
%         \item Designed a custom PCB featuring an ATSAMD21 MCU, 9-axis IMU, current sensor, motor, encoder, and SD
% interface
%         \item Wrote code in C++ to facilitate robotic actuation, real-time IMU sensor fusion, and data collection
%         \item Designed and fabricated models of the robot’s body and enclosure, using CAD software
%         \item Spoke at a symposium about this research project and presented findings at a poster presentation
%     \end{itemize}
% \end{minipage}
% \end{tabularx}

% \begin{tabularx}{\linewidth}{ @{}l r@{} }
% \textbf{Astrobotic Technology} - Pittsburgh, PA. \\[2pt]
% \textbf{Flat Sat Intern} \hfill \textbf{May 2023 - Aug. 2023} \\[2pt]
% \textit{Astrobotic is a space robotics company specializing in lunar payload delivery services and advanced space technologies} \\[2pt]
% \begin{minipage}[t]{\linewidth}
%     \begin{itemize}[nosep,after=\strut, leftmargin=1em, itemsep=2pt]
%         \item Collaborated with a cross-functional team to design and construct essential hardware for testing spacecraft performance
%         \item Built racks and harnesses to assist in the integration of and test on a hardware-in-the-loop platform
%         \item Designed a PCB featuring four distinct heat emulation designs, and conducted a comprehensive trade study comparing each
% design in terms of efficiency, heat output, complexity, and cost-effectiveness
%         \item Developed an embedded software driver to facilitate TCP packet transmission through Ethernet from a PC to the
% ”backplane,” enabling seamless data collection from the emulation hardware
%     \end{itemize}
% \end{minipage}
% \end{tabularx}

\newenvironment{twocolentry}[2][]{
    \onecolentry
    \def\secondColumn{#2}
    \setcolumnwidth{\fill, 3.5 cm}
    \begin{paracol}{2}
}{
    \switchcolumn \raggedleft \secondColumn
    \end{paracol}
    \endonecolentry
} % new environment for two column entries

\section{\textbf{EXPERIENCE}}
    \begin{twocolentry}{
        Philadelphia, PA \\
        July 2024 - Present}
        \textbf{DAIR Lab}, Research Assistant \\
        \textit{DAIR Lab develops algorithms for real-time planning and control of robots in contact-rich environments, using a model-based approach through perception modalities}
        \begin{itemize}[nosep, after=\strut, leftmargin=1em, itemsep=2pt]
            % \item Co-developed a \textbf{tactile sensing system} to infer object geometry, pose, and coefficient of friction based on a contact-implicit learning framework called ContactNets
            \item Contributed to the haptic exploration project, enabling robots to \textbf{perceive objects through touch}  
            \item Used a model-based approach to estimate the properties (e.g., geometry, trajectory, and coefficient of friction) by \textbf{minimizing the loss function} from our learning framework, ContactNets
            \item Researched \textbf{optimal model selection} that executes subsequent actions with the highest expected information gain (EIG)
            \item Implemented a force estimation algorithm for the DenseTact 2.0 optical tactile sensor 
            \item Validated methods using in simulation with \textbf{Drake} (robot physics sim) and hardware experiments
                % \item Collaborated on a {tactile sensing system}.
                % optimal model selection through observed information gain for unkown systems. 
                % \item Collaborated on a \textbf{tactile sensing system} for a multi-fingered robotic platform to execute optimal actions with maximum expected information gain
                % \item Developed a force estimation algorithm for the DenseTact 2.0 optical tactile sensor
                % \item Leveraged a learned dynamics model to estimate unknown object properties (e.g., friction, geometry, inertia) through optimal contact interactions
                % \item Developed a sim-to-real approach using Drake (robot physics simulator) and hardware experiments
            \end{itemize}
    \end{twocolentry}

    \begin{twocolentry}{
        Pittsburgh, PA \\
        May 2023 - Aug 2023}
        \textbf{Astrobotic Technology}, Hardware Development Intern \\
        \textit{Astrobotic is a space robotics company specializing in lunar payload delivery services and space tech}
        \begin{itemize}[nosep, after=\strut, leftmargin=1em, itemsep=2pt]
            \item Built essential hardware to assist in the integration of and test on a hardware-in-the-loop platform
            \item Designed a PCB featuring four distinct heat emulation designs and conducted a comprehensive trade study comparing each design in terms of efficiency, heat output, complexity, and cost
            \item Developed an embedded software driver to facilitate TCP packet transmission through Ethernet from a PC to the ”backplane,” enabling seamless data collection from the emulation hardware
            \end{itemize}
    \end{twocolentry}

    \begin{twocolentry}{
        Philadelphia, PA
        
        Feb 2021 - May 2023}
        \textbf{Sung Lab}, Research Assistant
        \begin{itemize}[nosep,after=\strut, leftmargin=1em, itemsep=2pt]
        \item Improved the design of a robotic swimmer that utilizes jet propulsion for locomotion
        \item Designed a custom PCB featuring an ATSAMD21 MCU, 9-axis IMU, current sensor, motor, encoder.
        \item Developed in C++ to facilitate robotic actuation, real-time IMU sensor fusion, and data collection
        \item Designed and fabricated models of the robot’s body and enclosure using CAD software
        \item Spoke at a symposium about this research and presented findings at a poster presentation
            \end{itemize}
    \end{twocolentry}